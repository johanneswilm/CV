
\opt{tea,ant}{
\section*{\Books}

\begin{itemize}
\item \textit{Nicaragua, Back from the Dead? An anthropological view of the Sandinista movement in the early 21st Century} ISBN: 978-82-8198-001-3, 340 \pages, \webaddress: \href{http://newleftnotes.org/nicaragua-back-from-the-dead}{http://newleftnotes.org/nicaragua-back-from-the-dead} Oslo \& Tucson: New Left Notes: \September 2011
    \begin{itemize}
    \item \publicpresentations \inside Managua, Tipitapa, León, Estelí, Tucson, Portland, Eugene
    \end{itemize}

\item \textit{On the Margins -- US Americans in a border town to Mexico} ISBN: 1-4116-6175-3, 236 \pages, \webaddress: \href{http://newleftnotes.org/on-the-margins}{http://newleftnotes.org/on-the-margins} Cornwall: Lulu Enterprises Inc.: \February 2006 (2nd edition 2011)
\end{itemize}
}

\section*{\peerreviewedscientificarticles}

\begin{itemize}
\opt{it}{
\item \textit{Opening and Reusing Transparent Peer Reviews with Automatic Article Annotation. Co-authors: Afshin Sadeghi, Sarven Capadisli, Philipp Mayr, Christoph Lange. Proceedings, Linked Research workshop at ESWC 2019.}
\item \textit{Decentralized creation of academic documents using a Network Attached Storage (NAS) server}. Co-authors: Afshin Sadeghi, Philipp Mayr, Christoph Lange. Proceedings, Linked Research workshop at ESWC 2017.
\item \textit{Opening Scholarly Communication in Social Sciences by Connecting Collaborative Authoring to Peer Review}. Co-authors: Afshin Sadeghi, Philipp Mayr, Christoph Lange. Information-Wissenschaft \& Praxis, \Volume 68 (2-3), 163-170, 2017.
\item \textit{Vivliostyle -- Open source, web browser based CSS typesetting engine}. Co-author: Shinyu Murakami. Balisage Series on Markup Technologies, Washington, DC, \USA, \Volume 15, \August 13, 2015.
\item \textit{Real-world challenges to collaborative text creation}. Co-author: Daniel Frebel. DChanges '14 Proceedings of the 2nd International Workshop on (Document) Changes: modeling, detection, storage and visualization. Article No. 8,  ACM New York, NY, USA, 2015.
}
\opt{tea,ant}{
\item \textit{On Sandinista Ideas of Past Connections to the Soviet Union and Nicaraguan Exceptionalism}. Chapter in "A Nicaraguan Exceptionalism? Debating the Legacy of the Sandinista Revolution", Institute of Latin America Studies, 2020.
\item \textit{The Significance of the Agrarian reform in Nicaragua}. Journal of Social and Development Sciences, \Volume 5, Issue 3, 2014.
\item \textit{Crossing the Line}. Sosialantropologistudentenes tidsskrift Betwixt and Between, \Volume 17, 2006.
\item \textit{Marxism \& Borders - Making the Case for the Relevance of Marxism in the Understanding of State Borders}. Sosialantropologistudentenes tidsskrift Betwixt and Between, \Volume 16, 2005.
}
\end{itemize}

\section*{\Other \conferencepapers}

\begin{itemize}
\opt{tea,ant}{
\item \textit{Honduras and Paraguay: The Short-Term Consequences of Successful Coups d'Etat Against Progressive Governments in Latin America in the 21st Century} \presentedat \textit{Radical Americas Symposium 2015}, UCL Institute of the Americas, London, \UK, \September 14--15, 2015.
}
\opt{it}{
\item \textit{Web browser based CSS typesetting engine -- How browser based typesetting systems can be made to work also for printed media}. Co-author: Shinyu Murakami. XML London 2015 Conference Proceedings, \UK, \June 6, 2015.
\item \textit{Annotation and dynamic web content: Why annotations need to be part of the source document} \presentedat \textit{Future for the Annotation of Digital Objects - workshop}, Leuphana University Lüneburg, \Germany, \May 12--13, 2015.
}
\opt{tea,ant}{
\item \textit{On Nicaraguan exceptionalism and the importance of past connections to the Soviet Union} \presentedat \textit{Frontiers in Central American Research}, Institute of Latin American Studies, School of Advanced Study, University of London, London, \UK, \March 9, 2015.
\item \textit{The interplay between ALBA and Sandinismo in current day Nicaragua} \presentedat \textit{Ten Years of the Bolivarian Alliance for the Peoples of Our America (ALBA): Progress, Problems, and Prospects}, Institute of Latin American Studies, School of Advanced Study, University of London, London, \UK, \February 26, 2015.
}
\opt{it}{
\item \textit{Workshop: the future of the book} \with Daniel Frebel, \presentedat \textit{Post-Digital Scholar Conference}, Leuphana University Lüneburg, \Germany, \November 12--14, 2014.
}
\end{itemize}

\section*{\Other \textpublications}

\begin{itemize}
\opt{tea,ant}{
\item \textit{Advokater og aktivister raser: »Bønder er idømt 120 års fængsel uden beviser«} (\english: Lawyers and activists in anger: »Farmers sent to 120 years of prison without evidence«), \href{http://modkraft.dk/artikel/advokater-og-aktivister-raser-b-nder-er-id-mt-120-rs-f-ngsel-uden-beviser}{http://modkraft.dk/artikel/advokater-og-aktivister-raser-b-nder-er-id-mt-120-rs-f-ngsel-uden-beviser}, Modkraft, \Denmark: \August 9, 2016.
\item \textit{Schatten des Staatsstreichs – Paraguay: Vier Jahre nach Putsch gegen Präsident Lugo müssen elf Bauern ins Gefängnis} (\english: Shadow of the coup d'etat – Paraguay: Four years after the coup against President Lugo, eleven farmers have to go to jail), \href{https://www.jungewelt.de/2016/08-02/027.php}{https://www.jungewelt.de/2016/08-02/027.php}, Junge Welt, Berlin, \Germany: \August 2, 2016.
\item \textit{Mysteriet om Curuguaty – den uløste massakren som felte presidenten} (\english: The mystery about Curuguaty – the unresolved mystery that made the President fall), \href{https://radikalportal.no/2016/03/30/mysteriet-om-curuguaty-den-uloste-massakren-som-felte-presidenten/}{https://radikalportal.no/2016/03/30/mysteriet-om-curuguaty-den-uloste-massakren-som-felte-presidenten/}, Radikal Portal, \Norway: \March 30, 2016.
}
\opt{it}{
\item \textit{Does Digital Publishing Need Standards? A History of Text File Standards and Outlook into the Future}, \href{https://research.consortium.io/docs/publishing_standards_dossier/publishing_standards_dossier.html\#second-subheading}{https://research.consortium.io/docs/publishing\_standards\_dossier/publishing\_standards\_dossier.html\#second-subheading}, Dossier: Standards in Digital Publishing - Practitioners' Viewpoints, Leuphana Universität Lüneburg, \Germany: \December, 2015.
}
\opt{tea,ant}{
\item \textit{Krisa til tross, NATO stiller seg mot kutt i Hellas militærbudsjett} (\english: Despite the crisis, NATO is opposed to cuts in Greece's military budget), \href{http://radikalportal.no/2015/07/02/krisa-til-tross-nato-stiller-seg-mot-kutt-i-hellas-militaerbudsjett/}{http://radikalportal.no/2015/07/02/krisa-til-tross-nato-stiller-seg-mot-kutt-i-hellas-militaerbudsjett/}, Radikal Portal, \Norway: \July 2, 2015.
\item \textit{Europas politikere har tenkt å fortsette å la båtflyktninger dø} (\english: The European politicians are planning to continue to let boat refugees die), \href{http://radikalportal.no/2015/04/23/europas-politikere-har-tenkt-a-fortsette-a-la-batflyktninger-do/}{http://radikalportal.no/2015/04/23/europas-politikere-har-tenkt-a-fortsette-a-la-batflyktninger-do/}, Radikal Portal, \Norway: \April 23, 2015.
\item \textit{En personlig (farget) rapport fra Ukraina} (\english: A personal (imprecise) account from Ukraine), \href{http://radikalportal.no/2014/12/17/en-personlig-farget-rapport-fra-ukraina/}{http://radikalportal.no/2014/12/17/en-personlig-farget-rapport-fra-ukraina/}, Radikal Portal, \Norway: \December 17, 2014.
}
\opt{it}{
\item \textit{7 Reasons We Shouldn't Let Ello Become the Next Facebook}, \href{http://wire.novaramedia.com/2014/10/7-reasons-we-shouldnt-let-bello-become-the-next-facebook/}{http://wire.novaramedia.com/2014/10/7-reasons-we-shouldnt-let-bello-become-the-next-facebook/}, Novaramedia, \UK: \October 6, 2014.
\item \textit{Intervju med Seth Schoen fra Electronic Frontier Foundation} (\english: Intervju with Seth Schoen from the Electronic Frontier Foundation), \href{http://radikalportal.no/2014/08/24/intervju-med-seth-schoen-fra-electronic-frontier-foundation/}{http://radikalportal.no/2014/08/24/intervju-med-seth-schoen-fra-electronic-frontier-foundation/}, Radikal Portal, \Norway: \August 24, 2014.
}
\opt{tea,ant}{
\item \textit{»Noch weit von einer Revolution entfernt« Brasilien: Proteste gegen Fahrpreiserhöhung haben sich laut Aktivisten zum Volksaufstand entwickelt. Ein Gespräch mit Vincente und Lorena Castillo} (\english: "Still far from a Revolution" Brazil: Protests against increments in bus prices have developed into a popular uprising according to activists. A conversation with Vincente and Lorena Castillo), \href{http://www.jungewelt.de/2013/07-25/036.php}{http://www.jungewelt.de/2013/07-25/036.php}, Junge Welt, Berlin, \Germany: \July 25, 2013.
\item \textit{Indianere i Chile kræver egen regering} (\english: Native Americans in Chile demand a government of their own), Flensborg Avis, Flensburg, \Germany: \July 20, 2013.
\item \textit{Vorsichtiger Optimismus in Paraguay} (\english: Cautious optimism in Paraguay), \href{http://www.jungewelt.de/2013/04-24/002.php}{http://www.jungewelt.de/2013/04-24/002.php}, Junge Welt, Berlin, \Germany: \April 24, 2013.
\item \textit{Gespaltene Linke} (\english: Divided Left), \href{http://www.jungewelt.de/2013/04-19/005.php}{http://www.jungewelt.de/2013/04-19/005.php}, Junge Welt, Berlin, \Germany: \April 19, 2013.
\item \textit{Latin-Amerika etter Hugo Chavez (1954-2013)} (\english: Latin America after Hugo Chavez (1954--2013)), \href{http://radikalportal.no/2013/03/06/latin-amerika-etter-hugo-chavez-1954-2013/}{http://radikalportal.no/2013/03/06/latin-amerika-etter-hugo-chavez-1954-2013/}, Radikal Portal, \Norway: \March 6, 2013.
\item \textit{Staatsstreich 2.0: Seit dem Putsch in Paraguay ist ein Monat vergangen. Ein Gespräch mit dem rechtmäßigen Präsidenten Fernando Lugo} (\english:Coup d'etat 2.0: A month has gone by since the coup d'etat. A conversation with the rightful president Fernando Lugo), Junge Welt, Berlin, \Germany: \July 23, 2012.
\item \textit{Post-Coup Paraguay: An Interview with Fernando Lugo}, \href{http://upsidedownworld.org/main/paraguay-archives-44/3766-post-coup-paraguay-an-interview-with-fernando-lugo}{http://upsidedownworld.org/main/paraguay-archives-44/3766-post-coup-paraguay-an-interview-with-fernando-lugo}, Upside Down World, \USA: \July 18, 2012.
    \begin{itemize}
    \item \translatedversion \malay \textit{Pasca Kudeta Di Paraguay: Sebuah Wawancara Dengan Fernando Lugo}, \href{http://www.berdikarionline.com/dunia-bergerak/20120719/pasca-kudeta-di-paraguay-wawancara-fernando-lugo.html}{http://www.berdikarionline.com/dunia-bergerak/20120719/pasca-kudeta-di-paraguay-wawancara-fernando-lugo.html}, Berdihari Online, \Indonesia: \July 19, 2012.
    \end{itemize}
\item \textit{--Fortsatt folkets leder: Paraguays nylig avsatte president Fernando Lugo i eksklusivt intervju til Klassekampen} (\english: --Still the people's leader: The newly unseated president of Paragua Fernando lugo in exclusive interview with Klassekampen), Klassekampen, Oslo, \Norway: \July 17 2012.
\item \textit{Putschisten ausgeschlossen: Paraguay -- Wachsende Proteste gegen Sturz von Präsident Lugo. UNASUR berät Lage nach dem Staatsstreich} (\english: Coupsters excluded: Paraguay -- Growing protests against overthrow of president Lugo. UNASUR discusses situation after coup d'etat), Junge Welt, Berlin, \Germany: \June 29, 2012.
\item \textit{Los campesinos toman el control de las protestas en Paraguay}, (\english: The campesinos take control of the protests in Paraguay), \href{http://www.el19digital.com/juventudpresidente/2012/06/los-campesinos-toman-el-control-de-las-protestas-en-paraguay/}{http://www.el19digital.com/juventudpresidente/2012/06/los-campesinos-toman-el-control-de-las-protestas-en-paraguay/}, Juventud Presidente, El 19 Digital, Managua, Nicaragua: \June 27, 2012.
\item \textit{Fragiles Bündnis: Paraguays gestürzter Präsident Fernando Lugo mußte gegen das Parlament um soziale Reformen kämpfen} (\english: Fragile coalition: The president of Paraguay who was recently removed, Fernando Lugo, had to fight against parliament for his social reforms) \with Michael Böhner, Junge Welt, Berlin, \Germany: \June 27, 2012.
\item \textit{Nach dem Staatsstreich: In Paraguay wird mit weiteren Demonstrationen gerechnet} (\english: After the Coup d'etat: In Paraguay, more demonstrations are expected), Junge Welt, Berlin, \Germany: \June 26, 2012.
\item \textit{Marsjerer mot kuppregime: Paraguays bønder på vei til hovedstaden for å kjempe mot avsettelsen av venstresidas president} (\english: Marching against the coup d'etat regime: The farmers of Paraguay on their way to the capital to fight against the unseating of the leftist president), Klassekampen, Oslo, \Norway: \June 26 2012.
\item \textit{Cuba: Economic changes and the future of socialism -- interview with Cuban professor José Bell Lara}, LINKS International Journal of Socialist Renewal, Sydney, \Australia: \September 2010.
    \begin{itemize}
    \item \translatedversion \turkish \textit{Neo-Liberalizm Küba’ya Giremeyecek}, \href{http://haber.sol.org.tr/dunyadan/neo-liberalizm-kuba-ya-giremeyecek-haberi-34202}{http://haber.sol.org.tr/dunyadan/neo-liberalizm-kuba-ya-giremeyecek-haberi-34202}, Sol Portal, \Turkey: \October 7 2010.
    \item \translatedversion \german \textit{Kuba wird nicht kapitalistisch}, Neues Deutschland, Berlin, \Germany: \November 9 2010.
    \item \translatedversion \norwegian \textit{Cuba etter reformene – fortsatt sosialistisk?}, Rødt!, Nr 4 2010, Oslo, \Norway: \December 2010.
    \end{itemize}
}
\item \textit{Datasystemet Albastryde} (\english: The Informationsystem "Albastryde"), Sosialistisk fremtid, Bergen, \Norway: \September 2010.
\opt{tea,ant}{
\item \textit{Der Widerstand in Honduras} (\english: The Resistance in Honduras), Rotdorn, Berlin, \Germany: \Spring 2010.
}
\item \textit{Nicaragua Builds An Innovative Agricultural Information System Using Open Source Software} Linux Journal, Houston, TX, \USA: \November 12 2009.
\opt{tea,ant}{
\item \textit{"Auch unpolitische fühlen jetzt was eine politische Diktatur ist"} Der Putsch vom 28. juni hat viele Hondureñer politisiert (\english: "Also non-political people now feel what a political dictatorship is like" The coup d'etat of June 28 has politicized many Hondurans) \with Michael Böhner, Junge Welt, Berlin, \Germany: \August 20, 2009.
\item \textit{Multikulturelle aspekter i norsk skole} (\english: Multicultural aspects in the Norwegian school) Under Utdanning, Oslo, \Norway: \September 2007.
\item \textit{Lærer i grænseland} (\english: Teacher in borderlands) Flensborg Avis, Flensburg, \Germany:  \November 11/13, 2006.
\item \textit{Lærer ved den USA-meksikanske grensen} (\english: Teacher near the the US-Mexican border) Under Utdanning, Oslo, \Norway: \December 2006.
\item \textit{Mexico's andre revolusjon -- En revolusjon "in real life"?} (\english: Mexico's second revolution -- a revolution "in real life"?) Tidsskriftet Argument, \Volume 3, University of Oslo, \Norway: \November 2006.
\item \textit{Den tvilsomme geografiske mobiliteten} (\english: The questionable geographical mobility) Tidsskriftet Argument, \Volume 2, University of Oslo, \Norway: \September 2006.

%\item \textit{24 minutes visual anthropology about (trans)nationalism on the Danish-German border Online video.} webaddress: http://antropologi.info/blog/anthropology/anthropology.php?p=1920 & http://savageminds.org/2006/06/28/border-nationalism-in-germany-an-ethnographic-film/ : \June 2006

\item \textit{Douglas: der USA opphører å eksistere} (\english: Douglas: where the USA ends to exist) Tidsskriftet Argument, \Volume 1, University of Oslo, \Norway: \June 2006.

\item \textit{Omgruppering på Blindern} (\english: Regroupment at Blindern) Kontur Debatt, \Volume 3, Oslo, \Norway: \May 2006. *
\item \textit{Defining the Territory. What Community do US Americans in a Border Town to Mexico belong to?} Master of Philosophy thesis, \December 2005, Institute of Social Anthropology, \uio.
\item \textit{IAEA ingen verdig fredsprisvinner} (\english: IAEA no worthy winner of Nobels peace prize) co-author, Aftenposten, Oslo, \Norway: \December 10, 2005.
\item \textit{Kontur Debatt og Venstresiden etter valget} (\english: Kontur Debate and the Left after the elections) co-author, Kontur debatt, \Volume 2, Oslo, \Norway: \October 2005.
\item \textit{Internasjonale erfaringer fra pensjonskampen i Tyskland} (\english: International experiences from the pension struggles in Germany) Kontur debatt, \Volume 1, Oslo, \Norway: \June 2005
\item \textit{En ø i USA} (\english: An Island in the USA) Flensborg Avis, Flensburg, \Germany: \October 22, 2004.

\item \textit{Netværk – demokrati – centralisme} (\english: Network -- Democracy -- Centralism) Socialistisk Information, \Volume 187, Copenhagen, \Denmark: \June 2004.

}

\end{itemize}

\opt{tea,ant}{
\section*{\documentaries}

\begin{itemize}
\item \textit{¿Qué pasó en Curuguaty? Un año después}, \with Ana Carolina Vera Resquín, \href{http://youtu.be/MoTHRzaarz8}{http://youtu.be/MoTHRzaarz8}, \June 2013.
\item \textit{Los estudiantes detrás de las marchas estudiantiles en Chile}, \href{http://vimeo.com/58220914}{http://vimeo.com/58220914}, \January 2013.
\item \textit{Las Vacas Colonas y la participación Sandinista}, \with Erick Rios, \December 2009.
\item \textit{La joven revolución hondureña}, \August/\September 2009.
    \begin{itemize}
    \item \publicpresentations \with \director \inside Managua, Tucson, Los Angeles (Human Rights Film Festival), Douglas, San Diego, Hermosillo (5 \numberoftimes), San Francisco, Oakland, Portland, Berlin (5 \numberoftimes), Lund, Copenhagen, Oslo (2 \numberoftimes), Bergen, London (4 \numberoftimes), Schleswig, Flensburg, Linkes Pfingstcamp Berlin/Brandenburg 2010, Solid Sommercamp 2010, Sofia.
    \item \publicpresentations \without \director \inside  Heidelberg, Mannheim, Granada, Stadt Hagen, Mexico City (\severaltimes), Embajada de Honduras \inside 12 \LatinAmericancountries.
    \item \radiointerviews \inside Radio Nueva Ya, Radio Nicaragua, Puente Sur (\severaltimes), Radio Bemba (\severaltimes), Nachrichtenpool Lateinamerika, Radio Nova, Honduras Contra el Golpe.
    \end{itemize}
\item \textit{Sugarcane burning in Chichigalpa}, Nicaragua, \March  2009.
\item \textit{9 year old girl working in Estelí}, Nicaragua, \February  2009.
\item \textit{Enfrentamiento FSLN-PLC León}, \November 2008.
\item \textit{Årsmøde Ascheffel}, \June 2006.
\end{itemize}

\section*{\Other \audiovisualpublications}

\begin{itemize}
\item \textit{Widerstand gegen Putschregierung: Cecilia Vuyk, Mitglied der Studentenbewegung, über Paraguays "kalten Putsch"} (\english: Resistance against coup government: Cecilia Vuyk, member of the student movement on the "cold coup d'etat" of Paraguay), \href{http://weltnetz.tv/video/370}{http://weltnetz.tv/video/370}, Weltnetz TV, Berlin, \Germany: \August 7, 2012.
\end{itemize}

\section*{\radioproductions}

\begin{itemize}
\item 2 \hours \show \with \music \und \interviews \inside \spanish \und \english, Sourcefabric Radio, \Czechrepublic, \every \Monday \und \Thursday, \September -- \November 2012.
\item \textit{La Historia del Movimiento Sandinista} (\english: The History of the Sandinista Movement), 50 minutes, Puente Sur Mexico, \May 17 2011.
\item \textit{Entrevista exclusiva con Fernando Lugo desde Asunción, Paraguay} (\english: Exclusive interview with Fernando Lugo from Asunción, Paraguay), Brigadas Puente Sur, Radio del Sur, \Mexico \und Venezuela: \July 18 2012.
\end{itemize}
}
