\section*{\ITworkexperience}

\begin{itemize}

\item \March 2017 – \December 2017: Scientific Employee, GESIS – Leibniz Institute for the Social Sciences

\opt{en}{Working on the Opening Scholarly Communication in the Social Sciences (OSCOSS) project.}

\item \April 2015 – : Editor of various editing related specifications, Editing Taskforce, World Wide Web Consortium (W3C)

\item \February 2015 – : Editor of Page Float spec, CSS Working Group, World Wide Web Consortium (W3C)

\item \February 2015 – \February 2017: Vivliostyle Inc.

\opt{en}{Technology evangelist, presenting Vivliostyle at a series of conferences such as XML London and the Frankfurt Bookfair about the advantages of using a tool such as Viviliostyle.js with CSS for layouting content into paged based media.}

\item \February 2014 – : WhyYuBuy LLC.

\opt{en}{Developer behind the iOS/Android application \textit{Klerify} which allows users to check grocery products for the types of certifications they have in terms of labor rights, sustainability, etc.}

\item \June 2014 – \May 2015: eHealth Africa

\opt{en}{Creation of various web based monitoring and control systems for organizations and institutions such as UNICEF and the United States Centers for Disease Control and Prevention (CDC), etc. .}

\item \June 2012 – \June 2013: Developer for Sourcefabric.org
\opt{en}{
Development of BookJS and collaboration with Booktype team on integration. BookJS is a Javscript library that can turn a webpage into a formatted book ready to print or create a PDF. It includes features such as footnotes, margin notes, cross references, table of figures, etc. .
}
\item \June 2012 –: Fidus Writer
\opt{en}{
Creation of an online WYSIWYG word processor for academic journals with citation management, formulas and footnotes as well as realtime collaboration. The editor is more than 80\% programmed in JavaScript, with the remaining parts being Python on the server. The JavaScript part relies among other things on the libraries jQuery, Underscore.js, Rangy, ICE, zip.js, Modernizr, jQuery UI.
}
\item \December 2011 – \March 2012: Consulting and Webprogramming for Fredslaget (\english: The Peace League)
\opt{en}{
Various campaign sites with creative features. JavaScript libraries used: AngularJS, EmberJS, jQuery, Backbone.js.
}

\item \September 2011 – \November 2011: Programming for PaperC.de
\opt{en}{
Creation of a web system for access management and page display of ebooks, including a Python backend and JavaScript frontend. The JavaScript part relies among other things on the libraries jQuery, Underscore.js, bootstrap, Backbone.js.
}

\item \Spring 2011 – \Spring 2012: Consulting and Webprogramming for ICAN Norway (Internationational Campaign to Abolish Nuclear Weapons)
\item \October 2010 – \November 2010: Google/Youtube \UK, London using Django, Python, Ajax/JavaScript, CSS/HTML.
\opt{en}{Included amongst other things the creation of a management systems of Google Coupons, and a theming system of the Youtube site for marketing purposes.}
\item \June 2010: JuWimm Group, Hannover using Ajax/JavaScript, CSS/HTML
\item \March 2010 – \May 2010: IT Developer Project Scio, Berlin using Python/Django/MySQL
\item \January 2009 – \September 2009: Senior IT Developer for Servicio de Información Mesoamericano sobre Agricultura Sostenible (SIMAS), Nicaragua, creating  Information System for Nicaraguan Ministry of Agriculture (MAGFOR) ALBAstryde
\item \May 2008: IT responsible Rødt party convention, Oslo
\item \April 2008 – \May 2008: Seminar series on Linux, Apache, MySQL, PHP (LAMP) and Joomla at Universidad de Managua en León, Nicaragua
\item \Spring 2008 – \Fall 2008: Construction, setup and promotion of websites \href{http://www.hentsoldatenehjem.org}{http://www.hentsoldatenehjem.org} (N) \href{http://www.hentsoldaternehjem.org}{http://www.hentsoldaternehjem.org} (DK) \href{http://www.stopwarinafghanistan.org}{http://www.stopwarinafghanistan.org} (worldwide)
\item \Spring 2008 – \Fall 2008: Patches for Gmerlin open source video editing framework
\item \Spring 2008: converting LatexLab.org editor from Windows/.Net C\# to Unix/Mono C\#
\item \Fall 2007 – \Spring 2008: Patches/translations for Latex bibliography package BiblaTeX

\item \Fall 2007 – : Owner of Wilm Web solutions http://www.wilm.no, including a number of new sites that are maintained (see wilm.no for details)
\item \January 2005 – \October 2009: Technically responsible for web pages of "Sosialantropologisk Forening" (\english: Social Anthropological Association)
\item 2005: Technically responsible for web pages of organization "Fredsinitiativet" \href{http://www.ingenkrig.no}{http://www.ingenkrig.no} (\english: The Peace Initiative)
\item 2005 – 2006: Technically and politically responsible for web pages of activist group "Blindern Fred" (\english: Blindern Peace) *
\item 2005 – 2006: Technically responsible and editorially co-responsible for web pages of student party "Venstrealliansen" \href{http://www.venstrealliansen.org}{http://www.venstrealliansen.org} (\english: Left Alliance)
\item \Spring 2005: Patches/translations to Scribus, open source DTP application
\item 2005: Technically responsible for \href{http://www.pensjonsomkamp.no}{http://www.pensjonsomkamp.no}
\item \Fall 2003: Programming of filter for Koffice and holder of KDE CVS account
\item \Spring 2003 – \Spring 2008: Programing of RV party convention software
\item 2002: Technically responsible for web pages of the Antiwar conference (offline)
\item \December 2002 – 2004: technically and editorially responsible for Indymedia Denmark (SF-active)
\item \September 2002: Technically responsible for "Euro Leftist Net"
\item \March 2002 – \August 2012: Technically main responsible for Indymedia Norway (first SF-active, later DadaIMC)
\item \February 2001 – \March 2005: Technically responsible for www.rv.no (Zope)
\item \December – \January 2000: Complete redesign of websites of the organization “Forum for Systemdebatt” (\english: Forum for system debate)
\item \Summer 2000: Collection of signatures from university employees in connecting with the Praha IMF-protests and 176 academic employees sign the petition “global solidaritet mot den globale kapitalismen” (\english: Global solidarity against global capitalism)
\item \Spring 2000: Media-stunt “danmark-til-ejderen.de”. The creation of the website is followed up by a series of posters in inner Flensburg.
\item \Spring 1999: IT-responsible for PDS Flensburg
\item \Spring 1998: Coworker in the Internet project “Aiken High School online”
\item \Fall 1997 – \Spring 1998: Founding of the website of my hometown Schuby
\item \Spring 1997: “Den Danske Skoleforening for Sydslesvig” (\english: Danish school organization of Southern Schleswig) as IT-worker (amongst other setting up a network to connect the organization's PCs)


\end{itemize}



*) "Blindern" = \uio


% \section*{\ITsystems}

% \begin{itemize}
% \item Linux (2.2– kernel, Linux Mint, Debian, Gentoo \und Ubuntu)
% \item Mac OS 9/X
% \item Windows 3.1–10
% \item FreeBSD/NetBSD
% \item OS 2 Warp, MS-DOS 6.2 \und BE OS
% \end{itemize}

\section*{\ITlanguages}

\begin{itemize}
\item JavaScript/Ajax (NCSA certified)
\item HTML 3.2/4.0/5.0 (NCSA certified)
\item XML 1.0/XHTML 1.0/XSL 1.0
\item CSS
\item SQL:  MySQL/PostgreSQL/Gadfly SQL
\item Python 2.X/3.X (Django)
% \item PHP
% \item Markup \languages: DTML, Facebook Markup Language (FBML)
\item \basic Java, Perl, C++, Basic
\end{itemize}

% \section*{\webframeworks}

% \begin{itemize}
% \item Django
% \item Grails
% \item Zope\translator{exchangeof}{utveksling av}
% \end{itemize}

% \section*{\hardware}

% \begin{itemize}
% \item \exchangeof: hard disk, optical disc drive, RAM, sound card, graphics card, BIOS, processor, network card, etc.
% \item network setup
% \end{itemize}
